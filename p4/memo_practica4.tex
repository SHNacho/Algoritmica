%%%%%%%%%%%%%%%%%%%%%%%%%%%%%%%%%%%%%%%%%
% Engineering Calculation Paper
% LaTeX Template
% Version 1.0 (20/1/13)
%
% This template has been downloaded from:
% http://www.LaTeXTemplates.com
%
% Original author:void fusion(int T[], int inicial, int final, int U[], int V[]){
% Dmitry Volynkin (dim_voly@yahoo.com.au)
%
% License:
% CC BY-NC-SA 3.0 (http://creativecommons.org/licenses/by-nc-sa/3.0/)
%
%%%%%%%%%%%%%%%%%%%%%%%%%%%%%%%%%%%%%%%%%

%----------------------------------------------------------------------------------------
%	PACKAGES AND OTHER DOCUMENT CONFIGURATIONS
%----------------------------------------------------------------------------------------

\documentclass[11pt,a4paper]{article} % Use A4 paper with a 12pt font size - different paper sizes will require manual recalculation of page margins and border positions

\usepackage[utf8]{inputenc}
\usepackage[spanish]{babel}
\usepackage{color}
\definecolor{gray97}{gray}{.97}
\definecolor{gray75}{gray}{.75}
\definecolor{gray45}{gray}{.45}
\usepackage{graphicx}
\usepackage{float}
\usepackage{listings}
\lstset{language=C++,
	frame=Ltb,
	framerule=0pt,
	aboveskip=.5cm,
	framextopmargin=3pt,
	framexbottommargin=3pt,
	framexleftmargin=0.2cm,
	framexrightmargin=0cm,
	framesep=0pt,
	rulesep=1.4pt,
	backgroundcolor=\color{gray97},
	rulesepcolor=\color{gray45},
	%
	stringstyle=\color{red!80!black}\ttfamily,
	showstringspaces = false,
	basicstyle=\ttfamily,
	commentstyle=\color{green!50!black}\ttfamily,
	keywordstyle=\color{blue}\ttfamily,
	morecomment=[l][\color{magenta}]{\#},
	%
	numbers=left,
	numbersep=10pt,
	numberstyle=\tiny,
	numberfirstline = false,
	breaklines=true,
	postbreak=\mbox{\textcolor{red}{$\hookrightarrow$}},
	xleftmargin=0.7cm,
	xrightmargin=0.3cm,
	tabsize=4,
}

% minimizar fragmentado de listados
\lstnewenvironment{listing}[1][]
{\lstset{#1}\pagebreak[0]}{\pagebreak[0]}

\lstdefinestyle{consola}
{basicstyle=\scriptsize\bf\ttfamily,
	backgroundcolor=\color{gray75},
}

\lstdefinestyle{C++}
{language=C++,
}


\usepackage{marginnote} % Required for margin notes
\usepackage{wallpaper} % Required to set each page to have a background
\usepackage{lastpage} % Required to print the total number of pages
\usepackage[left=1.3cm,right=4.6cm,top=2cm,bottom=4.0cm,marginparwidth=3.4cm]{geometry} % Adjust page margins
\usepackage{amsmath} % Required for equation customization
\usepackage{amssymb} % Required to include mathematical symbols
\usepackage{xcolor} % Required to specify colors by name
\usepackage{fancyhdr} % Required to customize headers
\setlength{\headheight}{80pt} % Increase the size of the header to accommodate meta-information
\pagestyle{fancy}\fancyhf{} % Use the custom header specified below
\renewcommand{\headrulewidth}{0pt} % Remove the default horizontal rule under the header

\setlength{\parindent}{0cm} % Remove paragraph indentation
\newcommand{\tab}{\hspace*{2em}} % Defines a new command for some horizontal space

\newcommand\BackgroundStructure{ % Command to specify the background of each page
\setlength{\unitlength}{1mm} % Set the unit length to millimeters

\setlength\fboxsep{0mm} % Adjusts the distance between the frameboxes and the borderlines
\setlength\fboxrule{0.5mm} % Increase the thickness of the border line
\put(10, 10){\fcolorbox{black}{blue!10}{\framebox(155,247){}}} % Main content box
\put(165, 10){\fcolorbox{black}{blue!10}{\framebox(37,247){}}} % Margin box
\put(10, 261){\fcolorbox{black}{white!10}{\framebox(192, 27){}}} % Header box
\put(145, 270){\includegraphics[height=15mm,keepaspectratio]{../logo.png}} % Logo box - maximum height/width: 
}

%----------------------------------------------------------------------------------------
%	HEADER INFORMATION
%----------------------------------------------------------------------------------------

\fancyhead[L]{\begin{tabular}{l r | l r} % The header is a table with 4 columns
		\textbf{ALGORÍTMICA} &  & \textbf{Adrián Carmona Lupiáñez} \\ % Project name and page count
		\textbf{Práctica} & 3 & \textbf{Ignacio Sánchez Herrera}  \\ % Version and reviewed date
		&  & \textbf{Jacobo Casado de Gracia} \\
		\textbf{Página} & \thepage/\pageref{LastPage} & \textbf{Jesús José Mª Maldonado Arroyo} \\
		& & \textbf{Juan Miguel Hernández Gómez} \\
\end{tabular}}

%----------------------------------------------------------------------------------------

\begin{document}

\AddToShipoutPicture{\BackgroundStructure} % Set the background of each page to that specified above in the header information section

%----------------------------------------------------------------------------------------
%	DOCUMENT CONTENT
%----------------------------------------------------------------------------------------


\title{\Huge \textbf{Subsecuencia en común más larga}}
\date{}
\maketitle
\thispagestyle{fancy}

\vspace{-2cm}

\section{Planteamiento del problema}
La Programación Dinámica se aplica en cuatro fases:
\begin{enumerate}
	\item Identificar la naturaleza n-etápica del problema
	\item Verificación del Principio de Optimalidad de Bellman
	\item Planteamiento de una recurrencia
	\item Cálculo de una solución.
\end{enumerate}

\textbf{Naturaleza n-etápica del problema}

	Sea la subsecuencia más larga $x_1, x_2, ..., x_n$; esta subsecuencia es resultado de una serie de sucesiones ya que tenemos que decidir los valores de $x_i$, con $1 \le i \le n$. Así primero tomaríamos una decisión sobre $x_1$, luego sobre $x_2$, y así sucesivamente.
	
	Por lo que podemos ver que estamos ante un problema de decisión n-etápico.\\

\textbf{Principio de Optimalidad de Bellman}

Si $x_i$ es un elemento intermedio de la subsecuencia más larga, entonces la subsecuencia $x_1, x_2, ..., x_i$ es una solución optimal, y también lo es la subsecuencia $x_i, x_{i+1}, ..., x_n$. \\

\textbf{Planteamiento de una recurrencia}

Sean $X_n$ la secuencia $x_1, x_2, \dots , x_n$ y $Y_m$ la secuencia $y_1, y_2, \dots , y_m$. Entonces llamemos $f(X_n, Y_m)$ al tamaño de la subsecuencia común más larga de $X_n$ y $Y_m$, donde:
$$
f(X_n, Y_m)= \left\{ \begin{array}{lcc}
	0 &   si  & n = 0, m = 0\\
	\\ 1 + f(X_{n-1}, Y_{m-1}) &  si & x_n = y_m \\
	\\ max(f(X_{n}, Y_{m-1}), f(X_{n-1}, Y_{m})) &  si  & x_n \neq y_m
	\end{array}
\right.
$$
\\
\textbf{Cálculo de una solución}

Para el cálculo de una solución crearemos una matriz en la que almacenaremos todos los valores posibles, evitando así operaciones repetidas, que en el caso de la recurrencia se repetirían.
Una vez obtenida la matriz, la reconstrucción de la subsecuencia común más larga es sencilla a partir de la recurrencia planteada.

Sea $m$ la matriz en la que se almacenarán los datos, en la que, la fila $i$ representa el valor $x_i$ y la columna $j$ el valor $y_j$:
$$
m(i, j)= \left\{ \begin{array}{lcc}
	0 &   si  & i = 0, j = 0\\
	\\ 1 + m(i-1, j-1) &  si & x_i = y_j \\
	\\ max(m(i-1,j), m(i,j-1)) &  si  & x_i \neq y_j
	\end{array}
\right.
$$
\\

\newpage
\section{Pseudocódigo}
Sean S1 y S2 las secuencias de las cuales queremos hallar la subsecuencia común más larga, el algoritmo se describiría por el siguiente pseudocódigo.

\small
\marginnote{\textbf{NOTA}: El algoritmo muestra al principio una función \textit{reconstrucción} que es recursiva y a la cual se la llama en primer lugar en la línea 24 del código mostrado.}
\normalsize 	
\begin{lstlisting}
INICIO DEL ALGORITMO
    cadena reconstruccion(matriz M, cadena a, cadena b, entero i, entero j)
    INICIO DE LA FUNCION
        Si i o j son 0:
            Devolver {}
        Si no, si a[i-1] es igual a b[j-1]:
            Devolver reconstruccion(M, a, b, i-1, j-1) + a[i-1]
        Si no, si M[i-1][j] es mayor que M[i][j-1]:
            Devolver reconstruccion(M, a, b, i-1, j)
        Si no:
            Devolver reconstruccion(M, a, b, i, j-1)
    FIN DE LA FUNCION
    
    TAM1=|S1|
    TAM2=|S2|
    Crear una matriz M con TAM1 filas y TAM2 columnas
    Rellenar la primera fila y columna de M con ceros
    Repetir desde i=1 hasta TAM1:
        Repetir desde j=1 hasta TAM2:
            Si S1[i-1] es igual a S2[j-1]:
                M[i][j]=M[i-1][j-1]+1
            Si no:
                M[i][j]=max(M[i-1][j],M[i][j-1])
    Devolver reconstruccion(M, S1, S2, TAM1, TAM2)
FIN DEL ALGORITMO

\end{lstlisting}


\newpage
\section{Eficiencia}
En cuanto a la eficiencia teórica del problema, tenemos dos cadenas de caracteres, una de tamaño $m$ y otra de tamaño $n$ (no tienen por qué ser iguales, como hemos visto a lo largo de toda la memoria).\\
Como vemos en el pseudocódigo primero se crea una matriz de tamaño $m \cdot n$, por lo que esta parte del algoritmo es de orden $O(n \cdot m)$.

Por otra parte la función de reconstrucción es una función recursiva. Sin embargo, como podemos ver, la reconstrucción se apoya en los valores de la matriz. Esto hace que la reconstrucción sea directa, por lo que esta parte del algoritmo es de orden $O(n)$.

Por lo tanto, podemos concluir que el algoritmo es de orden $O(n \cdot m)$.

\vspace{5mm}

En el caso de que esta matriz no existiese, estariamos hablando de un algoritmo de orden $O(2^n)$, ya que la función recursiva debería realizar todas las operaciones posibles, repitiendo muchas de ellas. Por lo tanto, la programación dinámica supone una gran mejora.



\newpage
\section{Código}
Aquí se muestra el código utilizado escrito en lenguaje C++.\\

Hemos indicado las 2 funciones utilizadas. La primera a la que se llama desde el \textit{main} es la función \textit{subsecuencia}. Y esta a su vez hace uso de la función \textit{reconstrucción}.

\begin{lstlisting}[style=C++]
//Funcion de reconstruccion recursiva
string reconstruccion(vector <vector <int>> matriz, string a, string b, int i, int j){
	if (i==0 || j==0)
		return "";
	else if (a[i-1] == b[j-1])
		return reconstruccion (matriz, a, b, i-1, j-1) + a[i-1];
	else if (matriz[i-1][j]>matriz[i][j-1])
		return reconstruccion (matriz, a, b, i-1, j);
	else
		return reconstruccion (matriz, a, b, i, j-1);
}

//Funcion para hallar la subsecuencia mas corta con programacion dinamica
string subsecuencia(string a, string b){
	int a_tam= a.size();
	int b_tam= b.size();
	
	//Creacion de la matriz con los valores y la matriz con las direcciones
	vector <vector <int>> matriz(a_tam+1, vector <int> (b_tam+1, 0));
	
	for (int i=1; i<a_tam+1; i++){
		for (int j=1; j<b_tam+1; j++){
			if (a[i-1]==b[j-1])
				matriz[i][j]= matriz[i-1][j-1]+1;
			else{
				if (matriz[i-1][j]>matriz[i][j-1])
					matriz[i][j]= matriz[i-1][j];
				else
					matriz[i][j]= matriz[i][j-1];
			}
		}
	}	
	
	//Return
	return reconstruccion(matriz, a, b, a_tam, b_tam);
}
\end{lstlisting}

\newpage
\section{Escenarios de ejecución}
Utilizando las secuencias \textit{jacobocasadodegracia} y \textit{jesusjosemariamaldonadoarroyo}, la subsecuencia común más larga es \textit{josadoaa}.\\

Aquí podemos ver la matriz de números que se construye.\\

\small

\hspace{1,72 em} \texttt{j e s u s j o s e m a r i a m a l d o n a d o a r r o y o}

\hspace{0,71 em} \texttt{0 0 0 0 0 0 0 0 0 0 0 0 0 0 0 0 0 0 0 0 0 0 0 0 0 0 0 0 0 0}

\texttt{j 0 1 1 1 1 1 1 1 1 1 1 1 1 1 1 1 1 1 1 1 1 1 1 1 1 1 1 1 1 1}

\texttt{a 0 1 1 1 1 1 1 1 1 1 1 2 2 2 2 2 2 2 2 2 2 2 2 2 2 2 2 2 2 2}

\texttt{c 0 1 1 1 1 1 1 1 1 1 1 2 2 2 2 2 2 2 2 2 2 2 2 2 2 2 2 2 2 2}

\texttt{o 0 1 1 1 1 1 1 2 2 2 2 2 2 2 2 2 2 2 2 3 3 3 3 3 3 3 3 3 3 3}

\texttt{b 0 1 1 1 1 1 1 2 2 2 2 2 2 2 2 2 2 2 2 3 3 3 3 3 3 3 3 3 3 3}

\texttt{o 0 1 1 1 1 1 1 2 2 2 2 2 2 2 2 2 2 2 2 3 3 3 3 4 4 4 4 4 4 4}

\texttt{c 0 1 1 1 1 1 1 2 2 2 2 2 2 2 2 2 2 2 2 3 3 3 3 4 4 4 4 4 4 4}

\texttt{a 0 1 1 1 1 1 1 2 2 2 2 3 3 3 3 3 3 3 3 3 3 4 4 4 5 5 5 5 5 5}

\texttt{s 0 1 1 2 2 2 2 2 3 3 3 3 3 3 3 3 3 3 3 3 3 4 4 4 5 5 5 5 5 5}

\texttt{a 0 1 1 2 2 2 2 2 3 3 3 4 4 4 4 4 4 4 4 4 4 4 4 4 5 5 5 5 5 5}

\texttt{d 0 1 1 2 2 2 2 2 3 3 3 4 4 4 4 4 4 4 5 5 5 5 5 5 5 5 5 5 5 5}

\texttt{o 0 1 1 2 2 2 2 3 3 3 3 4 4 4 4 4 4 4 5 6 6 6 6 6 6 6 6 6 6 6}

\texttt{d 0 1 1 2 2 2 2 3 3 3 3 4 4 4 4 4 4 4 5 6 6 6 7 7 7 7 7 7 7 7}

\texttt{e 0 1 2 2 2 2 2 3 3 4 4 4 4 4 4 4 4 4 5 6 6 6 7 7 7 7 7 7 7 7}

\texttt{g 0 1 2 2 2 2 2 3 3 4 4 4 4 4 4 4 4 4 5 6 6 6 7 7 7 7 7 7 7 7}

\texttt{r 0 1 2 2 2 2 2 3 3 4 4 4 5 5 5 5 5 5 5 6 6 6 7 7 7 8 8 8 8 8}

\texttt{a 0 1 2 2 2 2 2 3 3 4 4 5 5 5 6 6 6 6 6 6 6 7 7 7 8 8 8 8 8 8}

\texttt{c 0 1 2 2 2 2 2 3 3 4 4 5 5 5 6 6 6 6 6 6 6 7 7 7 8 8 8 8 8 8}

\texttt{i 0 1 2 2 2 2 2 3 3 4 4 5 5 6 6 6 6 6 6 6 6 7 7 7 8 8 8 8 8 8}

\texttt{a 0 1 2 2 2 2 2 3 3 4 4 5 5 6 7 7 7 7 7 7 7 7 7 7 8 8 8 8 8 8} \\

\normalsize

Y aquí podemos ver la matriz de direcciones que se construye.\\

\small

\hspace{1,72 em} \texttt{j e s u s j o s e m a r i a m a l d o n a d o a r r o y o}

\hspace{0,71 em} \texttt{0 0 0 0 0 0 0 0 0 0 0 0 0 0 0 0 0 0 0 0 0 0 0 0 0 0 0 0 0 0}

\texttt{j 0 \textbackslash{ }- - - - \textbackslash{ }- - - - - - - - - - - - - - - - - - - - - - -}

\texttt{a 0 | - - - - - - - - - \textbackslash{ }- - \textbackslash{ }- \textbackslash{ }- - - - \textbackslash{ }- - \textbackslash{ }- - - - -}

\texttt{c 0 | - - - - - - - - - | - - - - - - - - - - - - - - - - - -}

\texttt{o 0 | - - - - - \textbackslash{ }- - - - - - - - - - - \textbackslash{ }- - - \textbackslash{ }- - - \textbackslash{ }- \textbackslash}

\texttt{b 0 | - - - - - | - - - - - - - - - - - | - - - - - - - - - -}

\texttt{o 0 | - - - - - \textbackslash{ }- - - - - - - - - - - \textbackslash{ }- - - \textbackslash{ }- - - \textbackslash{ }- \textbackslash}

\texttt{c 0 | - - - - - | - - - - - - - - - - - | - - - | - - - - - -}

\texttt{a 0 | - - - - - | - - - \textbackslash{ }- - \textbackslash{ }- \textbackslash{ }- - - - \textbackslash{ }- - \textbackslash{ }- - - - -}

\texttt{s 0 | - \textbackslash{ }- \textbackslash{ }- - \textbackslash{ }- - - - - - - - - - - - | - - | - - - - -}

\texttt{a 0 | - | - - - - | - - \textbackslash{ }- - \textbackslash{ }- \textbackslash{ }- - - - \textbackslash{ }- - \textbackslash{ }- - - - -}

\texttt{d 0 | - | - - - - | - - | - - - - - - \textbackslash{ }- - - \textbackslash{ }- - - - - - -}

\texttt{o 0 | - | - - - \textbackslash{ }- - - | - - - - - - | \textbackslash{ }- - - \textbackslash{ }- - - \textbackslash{ }- \textbackslash}

\texttt{d 0 | - | - - - | - - - | - - - - - - \textbackslash{ }| - - \textbackslash{ }- - - - - - -}

\texttt{e 0 | \textbackslash{ }- - - - | - \textbackslash{ }- - - - - - - - | | - - | - - - - - - -}

\texttt{g 0 | | - - - - | - | - - - - - - - - | | - - | - - - - - - -}

\texttt{r 0 | | - - - - | - | - - \textbackslash{ }- - - - - - | - - | - - \textbackslash{ }\textbackslash{ }- - -}

\texttt{a 0 | | - - - - | - | - \textbackslash{ }- - \textbackslash{ }- \textbackslash{ }- - - - \textbackslash{ }- - \textbackslash{ }- - - - -}

\texttt{c 0 | | - - - - | - | - | - - | - - - - - - | - - | - - - - -}

\texttt{i 0 | | - - - - | - | - | - \textbackslash{ }- - - - - - - | - - | - - - - -}

\texttt{a 0 | | - - - - | - | - \textbackslash{ }- | \textbackslash{ }- \textbackslash{ }- - - - \textbackslash{ }- - \textbackslash{ }- - - - -}

\normalsize


Vamos a observar otro ejemplo utilizando las secuencias de caracteres \textit{ignaciosanchezherrera} y \textit{juanmiguelhernandezgomez, cuya subsecuencia común más larga es \textit{ignaneze}}.\\

\small

\hspace{1,72 em} \texttt{j u a n m i g u e l h e r n a n d e z g o m e z}

\hspace{0,71 em} \texttt{0 0 0 0 0 0 0 0 0 0 0 0 0 0 0 0 0 0 0 0 0 0 0 0 0}

\texttt{i 0 0 0 0 0 0 1 1 1 1 1 1 1 1 1 1 1 1 1 1 1 1 1 1 1}

\texttt{g 0 0 0 0 0 0 1 2 2 2 2 2 2 2 2 2 2 2 2 2 2 2 2 2 2}

\texttt{n 0 0 0 0 1 1 1 2 2 2 2 2 2 2 3 3 3 3 3 3 3 3 3 3 3}

\texttt{a 0 0 0 1 1 1 1 2 2 2 2 2 2 2 3 4 4 4 4 4 4 4 4 4 4}

\texttt{c 0 0 0 1 1 1 1 2 2 2 2 2 2 2 3 4 4 4 4 4 4 4 4 4 4}

\texttt{i 0 0 0 1 1 1 2 2 2 2 2 2 2 2 3 4 4 4 4 4 4 4 4 4 4}

\texttt{o 0 0 0 1 1 1 2 2 2 2 2 2 2 2 3 4 4 4 4 4 4 5 5 5 5}

\texttt{s 0 0 0 1 1 1 2 2 2 2 2 2 2 2 3 4 4 4 4 4 4 5 5 5 5}

\texttt{a 0 0 0 1 1 1 2 2 2 2 2 2 2 2 3 4 4 4 4 4 4 5 5 5 5}

\texttt{n 0 0 0 1 2 2 2 2 2 2 2 2 2 2 3 4 5 5 5 5 5 5 5 5 5}

\texttt{c 0 0 0 1 2 2 2 2 2 2 2 2 2 2 3 4 5 5 5 5 5 5 5 5 5}

\texttt{h 0 0 0 1 2 2 2 2 2 2 2 3 3 3 3 4 5 5 5 5 5 5 5 5 5}

\texttt{e 0 0 0 1 2 2 2 2 2 3 3 3 4 4 4 4 5 5 6 6 6 6 6 6 6}

\texttt{z 0 0 0 1 2 2 2 2 2 3 3 3 4 4 4 4 5 5 6 7 7 7 7 7 7}

\texttt{h 0 0 0 1 2 2 2 2 2 3 3 4 4 4 4 4 5 5 6 7 7 7 7 7 7}

\texttt{e 0 0 0 1 2 2 2 2 2 3 3 4 5 5 5 5 5 5 6 7 7 7 7 8 8}

\texttt{r 0 0 0 1 2 2 2 2 2 3 3 4 5 6 6 6 6 6 6 7 7 7 7 8 8}

\texttt{r 0 0 0 1 2 2 2 2 2 3 3 4 5 6 6 6 6 6 6 7 7 7 7 8 8}

\texttt{e 0 0 0 1 2 2 2 2 2 3 3 4 5 6 6 6 6 6 7 7 7 7 7 8 8}

\texttt{r 0 0 0 1 2 2 2 2 2 3 3 4 5 6 6 6 6 6 7 7 7 7 7 8 8}

\texttt{a 0 0 0 1 2 2 2 2 2 3 3 4 5 6 6 7 7 7 7 7 7 7 7 8 8}\\

\normalsize


Y aquí podemos ver la matriz de direcciones que se construye.\\

\small

\hspace{1,72 em} \texttt{j u a n m i g u e l h e r n a n d e z g o m e z}

\hspace{0,71 em} \texttt{0 0 0 0 0 0 0 0 0 0 0 0 0 0 0 0 0 0 0 0 0 0 0 0 0}

\texttt{i 0 - - - - - \textbackslash{ }- - - - - - - - - - - - - - - - - -}

\texttt{g 0 - - - - - | \textbackslash{ }- - - - - - - - - - - - \textbackslash{ }- - - -}

\texttt{n 0 - - - \textbackslash{ }- - | - - - - - - \textbackslash{ }- \textbackslash{ }- - - - - - - -}

\texttt{a 0 - - \textbackslash{ }- - - | - - - - - - | \textbackslash{ }- - - - - - - - -}

\texttt{c 0 - - | - - - | - - - - - - | | - - - - - - - - -}

\texttt{i 0 - - | - - \textbackslash{ }- - - - - - - | | - - - - - - - - -}

\texttt{o 0 - - | - - | - - - - - - - | | - - - - - \textbackslash{ }- - -}

\texttt{s 0 - - | - - | - - - - - - - | | - - - - - | - - -}

\texttt{a 0 - - \textbackslash{ }- - | - - - - - - - | \textbackslash{ }- - - - - | - - -}

\texttt{n 0 - - | \textbackslash{ }- - - - - - - - - \textbackslash{ }| \textbackslash{ }- - - - - - - -}

\texttt{c 0 - - | | - - - - - - - - - | | | - - - - - - - -}

\texttt{h 0 - - | | - - - - - - \textbackslash{ }- - - | | - - - - - - - -}

\texttt{e 0 - - | | - - - - \textbackslash{ }- - \textbackslash{ }- - - | - \textbackslash{ }- - - - \textbackslash{ }-} 

\texttt{z 0 - - | | - - - - | - - | - - - | - | \textbackslash{ }- - - - \textbackslash{ }}

\texttt{h 0 - - | | - - - - | - \textbackslash{ }- - - - | - | | - - - - -}

\texttt{e 0 - - | | - - - - \textbackslash{ }- | \textbackslash{ }- - - - - \textbackslash{ }| - - - \textbackslash{ }-}

\texttt{r 0 - - | | - - - - | - | | \textbackslash{ }- - - - - | - - - | -}

\texttt{r 0 - - | | - - - - | - | | \textbackslash{ }- - - - - | - - - | -}

\texttt{e 0 - - | | - - - - \textbackslash{ }- | \textbackslash{ }| - - - - \textbackslash{ }- - - - \textbackslash{ }-}

\texttt{r 0 - - | | - - - - | - | | \textbackslash{ }- - - - | - - - - | -}

\texttt{a 0 - - \textbackslash{ }| - - - - | - | | | - \textbackslash{ }- - - - - - - | -}



\normalsize


%----------------------------------------------------------------------------------------

%\marginnote{\textbf{NOTA}: Algoritmo al que calcular la eficiencia.}

\end{document}