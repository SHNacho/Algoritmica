%%%%%%%%%%%%%%%%%%%%%%%%%%%%%%%%%%%%%%%%%
% Engineering Calculation Paper
% LaTeX Template
% Version 1.0 (20/1/13)
%
% This template has been downloaded from:
% http://www.LaTeXTemplates.com
%
% Original author:void fusion(int T[], int inicial, int final, int U[], int V[]){
% Dmitry Volynkin (dim_voly@yahoo.com.au)
%
% License:
% CC BY-NC-SA 3.0 (http://creativecommons.org/licenses/by-nc-sa/3.0/)
%
%%%%%%%%%%%%%%%%%%%%%%%%%%%%%%%%%%%%%%%%%

%----------------------------------------------------------------------------------------
%	PACKAGES AND OTHER DOCUMENT CONFIGURATIONS
%----------------------------------------------------------------------------------------

\documentclass[11pt,a4paper]{article} % Use A4 paper with a 12pt font size - different paper sizes will require manual recalculation of page margins and border positions

\usepackage[utf8]{inputenc}
\usepackage[spanish]{babel}
\usepackage{color}
\definecolor{gray97}{gray}{.97}
\definecolor{gray75}{gray}{.75}
\definecolor{gray45}{gray}{.45}
\usepackage{graphicx}
\usepackage{float}
\usepackage{listings}
\lstset{language=C++,
	frame=Ltb,
	framerule=0pt,
	aboveskip=.5cm,
	framextopmargin=3pt,
	framexbottommargin=3pt,
	framexleftmargin=0.2cm,
	framexrightmargin=0cm,
	framesep=0pt,
	rulesep=1.4pt,
	backgroundcolor=\color{gray97},
	rulesepcolor=\color{gray45},
	%
	stringstyle=\color{red!80!black}\ttfamily,
	showstringspaces = false,
	basicstyle=\ttfamily,
	commentstyle=\color{green!50!black}\ttfamily,
	keywordstyle=\color{blue}\ttfamily,
	morecomment=[l][\color{magenta}]{\#},
	%
	numbers=left,
	numbersep=10pt,
	numberstyle=\tiny,
	numberfirstline = false,
	breaklines=true,
	postbreak=\mbox{\textcolor{red}{$\hookrightarrow$}},
	xleftmargin=0.7cm,
	xrightmargin=0.3cm,
	tabsize=4,
}

% minimizar fragmentado de listados
\lstnewenvironment{listing}[1][]
{\lstset{#1}\pagebreak[0]}{\pagebreak[0]}

\lstdefinestyle{consola}
{basicstyle=\scriptsize\bf\ttfamily,
	backgroundcolor=\color{gray75},
}

\lstdefinestyle{C++}
{language=C++,
}


\usepackage{marginnote} % Required for margin notes
\usepackage{wallpaper} % Required to set each page to have a background
\usepackage{lastpage} % Required to print the total number of pages
\usepackage[left=1.3cm,right=4.6cm,top=2cm,bottom=4.0cm,marginparwidth=3.4cm]{geometry} % Adjust page margins
\usepackage{amsmath} % Required for equation customization
\usepackage{amssymb} % Required to include mathematical symbols
\usepackage{xcolor} % Required to specify colors by name
\usepackage{fancyhdr} % Required to customize headers
\setlength{\headheight}{80pt} % Increase the size of the header to accommodate meta-information
\pagestyle{fancy}\fancyhf{} % Use the custom header specified below
\renewcommand{\headrulewidth}{0pt} % Remove the default horizontal rule under the header

\setlength{\parindent}{0cm} % Remove paragraph indentation
\newcommand{\tab}{\hspace*{2em}} % Defines a new command for some horizontal space

\newcommand\BackgroundStructure{ % Command to specify the background of each page
\setlength{\unitlength}{1mm} % Set the unit length to millimeters

\setlength\fboxsep{0mm} % Adjusts the distance between the frameboxes and the borderlines
\setlength\fboxrule{0.5mm} % Increase the thickness of the border line
\put(10, 10){\fcolorbox{black}{blue!10}{\framebox(155,247){}}} % Main content box
\put(165, 10){\fcolorbox{black}{blue!10}{\framebox(37,247){}}} % Margin box
\put(10, 261){\fcolorbox{black}{white!10}{\framebox(192, 27){}}} % Header box
\put(145, 270){\includegraphics[height=15mm,keepaspectratio]{../logo.png}} % Logo box - maximum height/width: 
}

%----------------------------------------------------------------------------------------
%	HEADER INFORMATION
%----------------------------------------------------------------------------------------

\fancyhead[L]{\begin{tabular}{l r | l r} % The header is a table with 4 columns
		\textbf{ALGORÍTMICA} &  & \textbf{Adrián Carmona Lupiáñez} \\ % Project name and page count
		\textbf{Práctica} & 3 & \textbf{Ignacio Sánchez Herrera}  \\ % Version and reviewed date
		&  & \textbf{Jacobo Casado de Gracia} \\
		\textbf{Página} & \thepage/\pageref{LastPage} & \textbf{Jesús José Mª Maldonado Arroyo} \\
		& & \textbf{Juan Miguel Hernández Gómez} \\
\end{tabular}}

%----------------------------------------------------------------------------------------

\begin{document}

\AddToShipoutPicture{\BackgroundStructure} % Set the background of each page to that specified above in the header information section

%----------------------------------------------------------------------------------------
%	DOCUMENT CONTENT
%----------------------------------------------------------------------------------------

En esta práctica vamos a analizar el uso de los algoritmos “voraces” o “greedy”, algoritmos que seleccionan en cada momento lo mejor de entre un conjunto de candidatos, sin tener en cuenta lo ya hecho, para obtener una solución “rápida” al problema.

Vamos a tener dos problemas a los cuales vamos a aplicar esta manera de resolverlos y mediremos su eficiencia teórica.

Una vez diseñado el algoritmo, veremos los resultados de la ejecución y los compararemos con los resultados “óptimos”, generados tras resolver el problema de la menor manera posible.

Recordemos que los algoritmos greedy no aseguran generar soluciones optimales siempre; esta desventaja es una ventaja en problemas en los que es muy difícil alcanzar la solución óptima, apliquemos el algoritmo que apliquemos, como el problema que se propone a continuación. No obstante, veremos que los resultados, a pesar de no ser los óptimos, son bastante eficientes, así como el tiempo de ejecución del algoritmo.


\section{Problema común (Viajante de comercio)}

Como hemos comentado anteriormente, aplicar un algoritmo que nos dé el resultado más óptimo para este problema es bastante complicado y su tiempo de ejecución se incrementaría bastante.

Es por eso por lo que el enfoque Greedy es una manera eficiente de solucionar este problema, generando un resultado que no es el óptimo pero se acerca a ello.

El problema se resume en encontrar un circuito hamiltoniano para una serie de puntos, en este caso ciudades, de manera que se recorran todas ellas sin volver a pasar por ninguna, de manera que la distancia total entre estas ciudades, es decir, del circuito, sea la mínima (y así minimizamos el recorrido).

\subsection{Algoritmo basado en cercanía}

\begin{lstlisting}[style=C++]

\end{lstlisting} 

\newpage

%----------------------------------------------------------------------------------------

%\marginnote{\textbf{NOTA}: Algoritmo al que calcular la eficiencia.}

\end{document}